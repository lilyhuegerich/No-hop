%
% API Documentation for Make Topology
% Module make_topology
%
% Generated by epydoc 3.0.1
% [Wed May  5 13:49:13 2021]
%

%%%%%%%%%%%%%%%%%%%%%%%%%%%%%%%%%%%%%%%%%%%%%%%%%%%%%%%%%%%%%%%%%%%%%%%%%%%
%%                          Module Description                           %%
%%%%%%%%%%%%%%%%%%%%%%%%%%%%%%%%%%%%%%%%%%%%%%%%%%%%%%%%%%%%%%%%%%%%%%%%%%%

    \index{make\_topology \textit{(module)}|(}
\section{Module make\_topology}

    \label{make_topology}

%%%%%%%%%%%%%%%%%%%%%%%%%%%%%%%%%%%%%%%%%%%%%%%%%%%%%%%%%%%%%%%%%%%%%%%%%%%
%%                               Functions                               %%
%%%%%%%%%%%%%%%%%%%%%%%%%%%%%%%%%%%%%%%%%%%%%%%%%%%%%%%%%%%%%%%%%%%%%%%%%%%

  \subsection{Functions}

    \label{make_topology:create_connected_ring}
    \index{make\_topology \textit{(module)}!make\_topology.create\_connected\_ring \textit{(function)}}

    \vspace{0.5ex}

\hspace{.8\funcindent}\begin{boxedminipage}{\funcwidth}

    \raggedright \textbf{create\_connected\_ring}(\textit{ring\_name}, \textit{level}, \textit{amount\_of\_switches}, \textit{topo}, \textit{hosts}, \textit{ip\_base}={\tt False}, \textit{switches}={\tt \texttt{[}\texttt{]}}, \textit{IDs}={\tt \texttt{[}\texttt{]}}, \textit{classic}={\tt True})

    \vspace{-1.5ex}

    \rule{\textwidth}{0.5\fboxrule}
\setlength{\parskip}{2ex}
    create a ring of amount\_of\_switches may switches with hosts many 
    swithes, switches are connected in a ring, returns ring object

\setlength{\parskip}{1ex}
    \end{boxedminipage}

    \label{make_topology:quick_test}
    \index{make\_topology \textit{(module)}!make\_topology.quick\_test \textit{(function)}}

    \vspace{0.5ex}

\hspace{.8\funcindent}\begin{boxedminipage}{\funcwidth}

    \raggedright \textbf{quick\_test}()

    \vspace{-1.5ex}

    \rule{\textwidth}{0.5\fboxrule}
\setlength{\parskip}{2ex}
    simple topology for quick testing

\setlength{\parskip}{1ex}
    \end{boxedminipage}

    \label{make_topology:test_ring}
    \index{make\_topology \textit{(module)}!make\_topology.test\_ring \textit{(function)}}

    \vspace{0.5ex}

\hspace{.8\funcindent}\begin{boxedminipage}{\funcwidth}

    \raggedright \textbf{test\_ring}(\textit{failover\_test}={\tt False})

\setlength{\parskip}{2ex}
\setlength{\parskip}{1ex}
    \end{boxedminipage}

    \label{make_topology:tree_topo}
    \index{make\_topology \textit{(module)}!make\_topology.tree\_topo \textit{(function)}}

    \vspace{0.5ex}

\hspace{.8\funcindent}\begin{boxedminipage}{\funcwidth}

    \raggedright \textbf{tree\_topo}(\textit{failover\_test}={\tt False})

    \vspace{-1.5ex}

    \rule{\textwidth}{0.5\fboxrule}
\setlength{\parskip}{2ex}
    tree topology for testing classic data center structures

\setlength{\parskip}{1ex}
    \end{boxedminipage}

    \label{make_topology:basic_3_ring}
    \index{make\_topology \textit{(module)}!make\_topology.basic\_3\_ring \textit{(function)}}

    \vspace{0.5ex}

\hspace{.8\funcindent}\begin{boxedminipage}{\funcwidth}

    \raggedright \textbf{basic\_3\_ring}(\textit{level\_zero\_amount\_of\_switches}={\tt 4}, \textit{level\_one\_amount\_of\_switches}={\tt 2}, \textit{outsIDe\_nodes}={\tt 1})

    \vspace{-1.5ex}

    \rule{\textwidth}{0.5\fboxrule}
\setlength{\parskip}{2ex}
    return ring\_v0\_1, ring\_v1\_1, ring\_v1\_2

\setlength{\parskip}{1ex}
    \end{boxedminipage}


%%%%%%%%%%%%%%%%%%%%%%%%%%%%%%%%%%%%%%%%%%%%%%%%%%%%%%%%%%%%%%%%%%%%%%%%%%%
%%                               Variables                               %%
%%%%%%%%%%%%%%%%%%%%%%%%%%%%%%%%%%%%%%%%%%%%%%%%%%%%%%%%%%%%%%%%%%%%%%%%%%%

  \subsection{Variables}

    \vspace{-1cm}
\hspace{\varindent}\begin{longtable}{|p{\varnamewidth}|p{\vardescrwidth}|l}
\cline{1-2}
\cline{1-2} \centering \textbf{Name} & \centering \textbf{Description}& \\
\cline{1-2}
\endhead\cline{1-2}\multicolumn{3}{r}{\small\textit{continued on next page}}\\\endfoot\cline{1-2}
\endlastfoot\raggedright \_\-\_\-p\-a\-c\-k\-a\-g\-e\-\_\-\_\- & \raggedright \textbf{Value:} 
{\tt None}&\\
\cline{1-2}
\end{longtable}


%%%%%%%%%%%%%%%%%%%%%%%%%%%%%%%%%%%%%%%%%%%%%%%%%%%%%%%%%%%%%%%%%%%%%%%%%%%
%%                           Class Description                           %%
%%%%%%%%%%%%%%%%%%%%%%%%%%%%%%%%%%%%%%%%%%%%%%%%%%%%%%%%%%%%%%%%%%%%%%%%%%%

    \index{make\_topology \textit{(module)}!make\_topology.configurationError \textit{(class)}|(}
\subsection{Class configurationError}

    \label{make_topology:configurationError}
\begin{tabular}{cccccccccc}
% Line for object, linespec=[False, False, False]
\multicolumn{2}{r}{\settowidth{\BCL}{object}\multirow{2}{\BCL}{object}}
&&
&&
&&
  \\\cline{3-3}
  &&\multicolumn{1}{c|}{}
&&
&&
&&
  \\
% Line for exceptions.BaseException, linespec=[False, False]
\multicolumn{4}{r}{\settowidth{\BCL}{exceptions.BaseException}\multirow{2}{\BCL}{exceptions.BaseException}}
&&
&&
  \\\cline{5-5}
  &&&&\multicolumn{1}{c|}{}
&&
&&
  \\
% Line for exceptions.Exception, linespec=[False]
\multicolumn{6}{r}{\settowidth{\BCL}{exceptions.Exception}\multirow{2}{\BCL}{exceptions.Exception}}
&&
  \\\cline{7-7}
  &&&&&&\multicolumn{1}{c|}{}
&&
  \\
&&&&&&\multicolumn{2}{l}{\textbf{make\_topology.configurationError}}
\end{tabular}


%%%%%%%%%%%%%%%%%%%%%%%%%%%%%%%%%%%%%%%%%%%%%%%%%%%%%%%%%%%%%%%%%%%%%%%%%%%
%%                                Methods                                %%
%%%%%%%%%%%%%%%%%%%%%%%%%%%%%%%%%%%%%%%%%%%%%%%%%%%%%%%%%%%%%%%%%%%%%%%%%%%

  \subsubsection{Methods}


\large{\textbf{\textit{Inherited from exceptions.Exception}}}

\begin{quote}
\_\_init\_\_(), \_\_new\_\_()
\end{quote}

\large{\textbf{\textit{Inherited from exceptions.BaseException}}}

\begin{quote}
\_\_delattr\_\_(), \_\_getattribute\_\_(), \_\_getitem\_\_(), \_\_getslice\_\_(), \_\_reduce\_\_(), \_\_repr\_\_(), \_\_setattr\_\_(), \_\_setstate\_\_(), \_\_str\_\_(), \_\_unicode\_\_()
\end{quote}

\large{\textbf{\textit{Inherited from object}}}

\begin{quote}
\_\_format\_\_(), \_\_hash\_\_(), \_\_reduce\_ex\_\_(), \_\_sizeof\_\_(), \_\_subclasshook\_\_()
\end{quote}

%%%%%%%%%%%%%%%%%%%%%%%%%%%%%%%%%%%%%%%%%%%%%%%%%%%%%%%%%%%%%%%%%%%%%%%%%%%
%%                              Properties                               %%
%%%%%%%%%%%%%%%%%%%%%%%%%%%%%%%%%%%%%%%%%%%%%%%%%%%%%%%%%%%%%%%%%%%%%%%%%%%

  \subsubsection{Properties}

    \vspace{-1cm}
\hspace{\varindent}\begin{longtable}{|p{\varnamewidth}|p{\vardescrwidth}|l}
\cline{1-2}
\cline{1-2} \centering \textbf{Name} & \centering \textbf{Description}& \\
\cline{1-2}
\endhead\cline{1-2}\multicolumn{3}{r}{\small\textit{continued on next page}}\\\endfoot\cline{1-2}
\endlastfoot\multicolumn{2}{|l|}{\textit{Inherited from exceptions.BaseException}}\\
\multicolumn{2}{|p{\varwidth}|}{\raggedright args, message}\\
\cline{1-2}
\multicolumn{2}{|l|}{\textit{Inherited from object}}\\
\multicolumn{2}{|p{\varwidth}|}{\raggedright \_\_class\_\_}\\
\cline{1-2}
\end{longtable}

    \index{make\_topology \textit{(module)}!make\_topology.configurationError \textit{(class)}|)}

%%%%%%%%%%%%%%%%%%%%%%%%%%%%%%%%%%%%%%%%%%%%%%%%%%%%%%%%%%%%%%%%%%%%%%%%%%%
%%                           Class Description                           %%
%%%%%%%%%%%%%%%%%%%%%%%%%%%%%%%%%%%%%%%%%%%%%%%%%%%%%%%%%%%%%%%%%%%%%%%%%%%

    \index{make\_topology \textit{(module)}!make\_topology.topo\_tracker \textit{(class)}|(}
\subsection{Class topo\_tracker}

    \label{make_topology:topo_tracker}
Keeps track of generated topology


%%%%%%%%%%%%%%%%%%%%%%%%%%%%%%%%%%%%%%%%%%%%%%%%%%%%%%%%%%%%%%%%%%%%%%%%%%%
%%                                Methods                                %%
%%%%%%%%%%%%%%%%%%%%%%%%%%%%%%%%%%%%%%%%%%%%%%%%%%%%%%%%%%%%%%%%%%%%%%%%%%%

  \subsubsection{Methods}

    \label{make_topology:topo_tracker:__init__}
    \index{make\_topology \textit{(module)}!make\_topology.topo\_tracker \textit{(class)}!make\_topology.topo\_tracker.\_\_init\_\_ \textit{(method)}}

    \vspace{0.5ex}

\hspace{.8\funcindent}\begin{boxedminipage}{\funcwidth}

    \raggedright \textbf{\_\_init\_\_}(\textit{self}, \textit{file\_name}, \textit{file\_path}={\tt \texttt{'}\texttt{../basic/}\texttt{'}})

\setlength{\parskip}{2ex}
\setlength{\parskip}{1ex}
    \end{boxedminipage}

    \label{make_topology:topo_tracker:add_switches_to_topo}
    \index{make\_topology \textit{(module)}!make\_topology.topo\_tracker \textit{(class)}!make\_topology.topo\_tracker.add\_switches\_to\_topo \textit{(method)}}

    \vspace{0.5ex}

\hspace{.8\funcindent}\begin{boxedminipage}{\funcwidth}

    \raggedright \textbf{add\_switches\_to\_topo}(\textit{self}, \textit{switches}, \textit{fail}={\tt False})

    \vspace{-1.5ex}

    \rule{\textwidth}{0.5\fboxrule}
\setlength{\parskip}{2ex}
    adds generated switches to a already existing topo\_tracker fail=True 
    is purposfull failing of a link for testing purposes.

\setlength{\parskip}{1ex}
    \end{boxedminipage}

    \label{make_topology:topo_tracker:add_hosts_to_topo}
    \index{make\_topology \textit{(module)}!make\_topology.topo\_tracker \textit{(class)}!make\_topology.topo\_tracker.add\_hosts\_to\_topo \textit{(method)}}

    \vspace{0.5ex}

\hspace{.8\funcindent}\begin{boxedminipage}{\funcwidth}

    \raggedright \textbf{add\_hosts\_to\_topo}(\textit{self}, \textit{rings}, \textit{amount}, \textit{client}={\tt False}, \textit{connected\_switches}={\tt \texttt{[}\texttt{]}})

\setlength{\parskip}{2ex}
\setlength{\parskip}{1ex}
    \end{boxedminipage}

    \label{make_topology:topo_tracker:connect_nodes}
    \index{make\_topology \textit{(module)}!make\_topology.topo\_tracker \textit{(class)}!make\_topology.topo\_tracker.connect\_nodes \textit{(method)}}

    \vspace{0.5ex}

\hspace{.8\funcindent}\begin{boxedminipage}{\funcwidth}

    \raggedright \textbf{connect\_nodes}(\textit{self}, \textit{ring\_0\_nodes}, \textit{ring\_1\_nodes})

    \vspace{-1.5ex}

    \rule{\textwidth}{0.5\fboxrule}
\setlength{\parskip}{2ex}
    connects all nodes in ring 0 to all nodes in ring 1

\setlength{\parskip}{1ex}
    \end{boxedminipage}

    \label{make_topology:topo_tracker:formalize_table}
    \index{make\_topology \textit{(module)}!make\_topology.topo\_tracker \textit{(class)}!make\_topology.topo\_tracker.formalize\_table \textit{(method)}}

    \vspace{0.5ex}

\hspace{.8\funcindent}\begin{boxedminipage}{\funcwidth}

    \raggedright \textbf{formalize\_table}(\textit{self}, \textit{switch}, \textit{fail}={\tt False})

    \vspace{-1.5ex}

    \rule{\textwidth}{0.5\fboxrule}
\setlength{\parskip}{2ex}
    formalize table values to send to the software switch

\setlength{\parskip}{1ex}
    \end{boxedminipage}

    \label{make_topology:topo_tracker:generate_ipv4_lpm_table}
    \index{make\_topology \textit{(module)}!make\_topology.topo\_tracker \textit{(class)}!make\_topology.topo\_tracker.generate\_ipv4\_lpm\_table \textit{(method)}}

    \vspace{0.5ex}

\hspace{.8\funcindent}\begin{boxedminipage}{\funcwidth}

    \raggedright \textbf{generate\_ipv4\_lpm\_table}(\textit{self}, \textit{switch}, \textit{fail}={\tt False})

    \vspace{-1.5ex}

    \rule{\textwidth}{0.5\fboxrule}
\setlength{\parskip}{2ex}
    IPv4 tables for comparision of No\_hop to classic look up proccess

\setlength{\parskip}{1ex}
    \end{boxedminipage}

    \label{make_topology:topo_tracker:path_finder_ip}
    \index{make\_topology \textit{(module)}!make\_topology.topo\_tracker \textit{(class)}!make\_topology.topo\_tracker.path\_finder\_ip \textit{(method)}}

    \vspace{0.5ex}

\hspace{.8\funcindent}\begin{boxedminipage}{\funcwidth}

    \raggedright \textbf{path\_finder\_ip}(\textit{self})

    \vspace{-1.5ex}

    \rule{\textwidth}{0.5\fboxrule}
\setlength{\parskip}{2ex}
    wrapper for P4 tutorial shortes path function used in LPM tables

\setlength{\parskip}{1ex}
    \end{boxedminipage}

    \label{make_topology:topo_tracker:check_links}
    \index{make\_topology \textit{(module)}!make\_topology.topo\_tracker \textit{(class)}!make\_topology.topo\_tracker.check\_links \textit{(method)}}

    \vspace{0.5ex}

\hspace{.8\funcindent}\begin{boxedminipage}{\funcwidth}

    \raggedright \textbf{check\_links}(\textit{self})

    \vspace{-1.5ex}

    \rule{\textwidth}{0.5\fboxrule}
\setlength{\parskip}{2ex}
    Verfiy correct link configuration , especially checking for duplicates

\setlength{\parskip}{1ex}
    \end{boxedminipage}

    \label{make_topology:topo_tracker:create_json}
    \index{make\_topology \textit{(module)}!make\_topology.topo\_tracker \textit{(class)}!make\_topology.topo\_tracker.create\_json \textit{(method)}}

    \vspace{0.5ex}

\hspace{.8\funcindent}\begin{boxedminipage}{\funcwidth}

    \raggedright \textbf{create\_json}(\textit{self})

    \vspace{-1.5ex}

    \rule{\textwidth}{0.5\fboxrule}
\setlength{\parskip}{2ex}
    print topo object to JSON

\setlength{\parskip}{1ex}
    \end{boxedminipage}

    \index{make\_topology \textit{(module)}!make\_topology.topo\_tracker \textit{(class)}|)}
    \index{make\_topology \textit{(module)}|)}
